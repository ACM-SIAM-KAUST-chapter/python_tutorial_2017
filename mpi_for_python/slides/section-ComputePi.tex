\begin{frame}
  \frametitle{Compute Pi}
  \Large
  \begin{equation*}
    \pi =
    \int_0^1 \frac{4}{1+x^2} dx \approx
    \frac{1}{n}\sum_{i=0}^{n-1}\frac{4}{1+(\frac{i+0.5}{n})^2}
  \end{equation*}
\end{frame}

\begin{frame}[t]
  \frametitle{Compute Pi -- sequential}
  \inputminted[linenos]{python}{compute_pi-seq.py}
\end{frame}

\begin{frame}[t]
  \frametitle{Compute Pi -- parallel [1]}
  \inputminted[linenos,firstline=1,lastline=14]{python}
  {compute_pi-mpi.py}
\end{frame}

\begin{frame}[t]
  \frametitle{Compute Pi -- parallel [2]}
  \inputminted[linenos,firstline=16]{python}
  {compute_pi-mpi.py}
\end{frame}

\begin{frame}
  \frametitle{Exercise \#4}
  Modify \emph{Compute Pi} example to employ NumPy.\\
  \smallskip\smallskip
  \textbf{Tip}: you can convert a Python \texttt{int}/\texttt{float}
  object to a NumPy \emph{scalar} with \texttt{x~=~numpy.array(x)}.
\end{frame}
